\begin{functional}
        \setcounter{enumi}{5}
        
        \item \textbf{Mostrar una tabla con estadísticas:} La aplicación debe de proveer al usuario de una tabla con información sobre el tráfico, en concreto, la dirección IP fuente y destino, y, los paquetes totales, paquetes por segundo, KBytes totales y Mbits por segundo de cada tupla. Esta tabla estará paginada con el fin de no mostrar excesiva información.
        \begin{functional}
        \item \textbf{Búsqueda en la tabla:} En la tabla se podrá buscar por dirección IP de destino o fuente. 
        \item \textbf{Ordenación de la tabla:} Por defecto la tabla estará ordenada por Kbytes pero se podrá ordenar por los paquetes por segundo, Mbits/s y paquetes totales.
        \item \textbf{Guardar la información de la tabla :} El usuario podrá descargar la información de la tabla en formato CSV para tener una huella del tráfico de la máquina y las direcciones entrantes y salientes.
        \item \textbf{Actualizar la tabla :} El sistema tendrá un botón para poder actualizar la tabla.
        \item \textbf{Actualizar la tabla automáticamente:} El usuario podrá marcar la opción que de los datos de la tabla se actualicen automáticamente.
        \item \textbf{Vaciar las estadísticas :} Con el fin de proporcionar legibilidad la aplicación contará con un sistema que permitirá vaciar las estadísticas para borrar los registros antiguos.
        \end{functional}
        \item \textbf{Recoger trazas:} El sistema deberá recoger una traza de los últimos 60 segundos de tráfico en formato PCAP. 
\end{functional}
