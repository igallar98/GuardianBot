El subsistema de autenticación es el menos extenso de la aplicación pero uno de los más importantes ya que engloba la funcionalidad referente al manejo de la sesión, y autenticación en la API REST. Permite al usuario o programa acceder a la aplicación.
Por una parte, inicio de sesión se realiza mediante una contraseña maestra guardada con HASH+SALT que se configura la primera vez que se instala la aplicación. En caso de pérdida o cambio de contraseña es necesario hacer una reinstalación del cortafuegos. Por otra, en la API se realiza mediante una llave que puede regenerar el usuario.
Finalmente, se podrá cerrar la sesión e incluso apagar el cortafuegos y todos sus componentes liberándolos de forma correcta.
