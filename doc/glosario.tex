\begin{description}
    \item [BPF] El filtro de paquetes de Berkeley es una tecnología utilizada en ciertos sistemas operativos para los programas que necesitan, entre otras cosas, analizar el tráfico de red. Permite enviar, recibir y filtrar paquetes de la capa de enlace sin procesar.
    \item [eBPF] Se trata de la versión extendida de BPF que, entre otras cosas, proporciona un intérprete de usuario para BPF con la implementación de nuevas librerías y los mapas BPF.
    \item [Clang] Capa de presentación de compilador un para los lenguajes de programación C, C++, Objective-C y Objective-C++.
     \item [LLVM ] Es la arquitectura interna de un compilador para los lenguajes de programación C, C++, Objective-C y Objective-C++.
   \item [ELF] Fichero que es producto de la compilación de un programa BPF con clang.
  \item [XDP] Es una ruta de datos de alto rendimiento basada en eBPF.
  \item [DOS] Un ataque de denegación del servicio (en inglés DOS) es un tipo de ataque que causa que una máquina o servicio sea inaccesible para los usuarios legítimos.
  \item [DDOS] Es un ataque de denegación del servicio pero distribuido desde un gran número de ordenadores. Mucho más difícil de detener y cuyo impacto es mayor.
   \item [NIC] Se trata de la tarjeta de red, también conocida como interfaz de red física.
  \item [Netfilter] Es una comunidad de desarrolladores de software que han creado un marco de trabajo disponible en el núcleo de Linux que permite interceptar y manipular paquetes de red.
  \item [Iptables] Es una utilidad que permite dar instrucciones al cortafuegos del núcleo de Linux.
    \item [VPS] Se trata de un servidor privado virtual cuyo hardware puede ser compartido.
    \item [PCAP] Es el formato de un fichero con una captura de paquetes de red.
    \item [API] Permite la comunicación entre dos aplicaciones de software a través de un conjunto de reglas y protocolos.
    \item [API REST] Es una API que respeta el protocolo HTTP para su implementación.
  
\end{description}