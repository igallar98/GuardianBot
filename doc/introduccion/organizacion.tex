El presente documento tiene como propósito la descripción del plan de proyecto, con el fin de que el lector pueda conocer las diferentes fases en las que se ha llevado a cabo el proyecto y el motivo de las decisiones que se han tomado para crear el producto final.
En el documento se pueden distinguir los siguientes capítulos:
\begin{itemize}
\item \textbf{Estado de arte:}  En esta sección se hablará sobre las soluciones que existen actualmente en el mercado mostrando las diferencias, similitudes, ventajas y desventajas con respecto a la que será desarrollada.
\item \textbf{Diseño:} Dentro de este capítulo se describirán con detalle los objetivos de la aplicación y las características generales del sistema por subsistemas para una mejor comprensión. También se detallarán los requisitos funcionales y no funcionales que debe cumplir la aplicación y, por último, se mostrarán las pantallas de la aplicación y cómo se interactúa con ellas.
\item \textbf{Desarrollo:} En este apartado se explicará cómo funciona la aplicación de forma interna, los problemas encontrados y las soluciones planteadas.
\item \textbf{Integración, pruebas y resultados: }En esta sección se mostrará la integración del cortafuegos en Linux y el uso de la interfaz en los navegadores, así como las pruebas realizadas a la aplicación.
\item \textbf{Conclusiones: }Por último, este apartado engloba las conclusiones del trabajo realizado y el trabajo futuro sobre este proyecto.
\end{itemize}
Por último, en los \textbf{apéndices} se proporcionará un manual de instalación de la aplicación y un manual para el programador. También estará disponible un cronograma con la planificación del proyecto, el GitHub y una demostración en tiempo real con ejemplos. 