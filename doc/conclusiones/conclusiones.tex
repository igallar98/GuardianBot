En este capítulo de la memoria se darán unas conclusiones sobre el aprendizaje que ha supuesto este proyecto y un planteamiento de trabajo futuro para mejorar la aplicación creada.

\paragraph{Conclusiones}
La principal idea de este proyecto ha sido crear un cortafuegos muy sencillo de utilizar para el usuario y con la capacidad de que un desarrollador pueda incorporar sus propias huellas y configuraciones fácilmente. Esto puede ser complejo ya que hay que tener especial cuidado con el código que se inserta en el núcleo, una parte crítica del sistema, y se le añade la dificultad de comunicar la aplicación del núcleo con la del espacio usuario y con la interfaz escrita en un lenguaje de alto nivel.
\\La elaboración de este proyecto me ha permitido apreciar las diferentes fases de un proyecto de programación, como la investigación, el diseño y la metodología de desarrollo para construir una aplicación funcional en el menor tiempo posible. También he comprendido la importancia del uso de una metodología ágil con el fin de mejorar la calidad y productividad dividiendo el proyecto en pequeñas partes.
\\Esta aplicación combina el bajo nivel con el alto nivel y usa tecnologías novedosas, como XDP, que hace que los errores no sean tan fáciles de resolver debido al pequeño tamaño de la comunidad de desarrolladores y las constantes actualizaciones al tratarse de una tecnología nueva.
\\Por último, se ha construido una aplicación cortafuegos para Linux completamente funcional y que cumple con todos los requisitos funcionales y no funcionales. Esta aplicación es capaz de utilizar una tecnología novedosa a bajo nivel, en el núcleo de Linux y, utilizando el alto nivel proporciona una interfaz sencilla para el usuario y facilita el desarrollo de soluciones de otros programadores mediante el uso de una API REST. El diseño de la interfaz se adapta al tamaño de la pantalla.

\paragraph{Trabajo futuro}
En cuanto al trabajo futuro se pretende traducir la aplicación a otros idiomas, como el inglés, con el fin de que pueda ser utilizada por más personas. Por otro lado, se quiere añadir al repositorio GitHub las huellas y soluciones aportadas por otros desarrolladores para que puedan ser utilizadas por toda la comunidad. También es posible mejorar la aplicación creando nuevas soluciones de detección de huellas, actualizándola y mejorando su compatibilidad con todos los sistemas operativos.